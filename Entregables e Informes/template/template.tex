%%%%%%%%%%%%%%%%%%%%%%%%%%%%%%%%%%%%%%%%%
% Simple Sectioned Essay Template
% LaTeX Template
%
% This template has been downloaded from:
% http://www.latextemplates.com
%
% Note:
% The \lipsum[#] commands throughout this template generate dummy text
% to fill the template out. These commands should all be removed when 
% writing essay content.
%
%%%%%%%%%%%%%%%%%%%%%%%%%%%%%%%%%%%%%%%%%

%----------------------------------------------------------------------------------------
%	PACKAGES AND OTHER DOCUMENT CONFIGURATIONS
%----------------------------------------------------------------------------------------

\documentclass[12pt]{article} % Default font size is 12pt, it can be changed here

\usepackage{geometry} % Required to change the page size to A4
\geometry{a4paper} % Set the page size to be A4 as opposed to the default US Letter

\usepackage{graphicx} % Required for including pictures

\usepackage{float} % Allows putting an [H] in \begin{figure} to specify the exact location of the figure
\usepackage{wrapfig} % Allows in-line images such as the example fish picture

\usepackage{lipsum} % Used for inserting dummy 'Lorem ipsum' text into the template

\usepackage[utf8]{inputenc}
\usepackage[T1]{fontenc}
\usepackage{lmodern}
\usepackage{longtable} 
\usepackage{booktabs}

\usepackage[T3,T1]{fontenc}
\usepackage[english]{babel}
\usepackage[noenc]{tipa}
\usepackage{tipx}
\usepackage{pifont}
\usepackage{eurosym}
\usepackage{amsmath}
\usepackage{wasysym}
\usepackage{amssymb,amsfonts,textcomp}
\usepackage{color}
\usepackage{array}
\usepackage{supertabular}
\usepackage{hhline}
\usepackage{hyperref}
\hypersetup{pdftex, colorlinks=true, linkcolor=blue, citecolor=blue, filecolor=blue, urlcolor=blue, pdfauthor=, pdfsubject=, pdfkeywords=}
\makeatletter
\newcommand\arraybslash{\let\\\@arraycr}
\makeatother


\usepackage{titlesec}
\newcommand{\sectionbreak}{\clearpage}

%Header
\usepackage{graphicx}

\usepackage{fancyhdr}
\pagestyle{fancy}

\usepackage{xcolor}

\graphicspath{{Pictures/}} 


\usepackage{lipsum}
\setlength\headheight{16pt} %% just to make warning go away. Adjust the value after looking into the warning.
% \rhead{{\color{blue}\rule{1cm}{1cm}}}

\rhead{\includegraphics[width=1cm]{logo}\includegraphics[width=2.5cm]{logo_botbloq}}
\lhead{BOTBLOQ: Ecosistema integral para el diseño, fabricación\\y programación de robots DIY}

\lfoot{\center{\footnotesize { \hspace{1cm} \newline Proyecto Financiado por el Centro de Desarrollo Tecnológico Industrial (CDTI).\\Expediente IDI-20150289 \\ Cofinanciado por el Fondo Europeo de Desarrollo Regional (FEDER) a través del Programa Operativo Plurirregional de Crecimiento Inteligente 2014-2020} \center{\includegraphics[width=2cm]{CDTI}
\includegraphics[width=2cm]{FEDER}}}}

\linespread{1.2} % Line spacing

%\setlength\parindent{0pt} % Uncomment to remove all indentation from paragraphs

%\graphicspath{Pictures/} % Specifies the directory where pictures are stored

\begin{document}

%----------------------------------------------------------------------------------------
%	TITLE PAGE
%----------------------------------------------------------------------------------------

\begin{titlepage}

\newcommand{\HRule}{\rule{\linewidth}{0.5mm}} % Defines a new command for the horizontal lines, change thickness here

\center % Center everything on the page

\includegraphics[width=2.5cm]{logo}
\includegraphics[width=6cm]{logo_botbloq}\\[0.5cm]

{ \huge \bfseries BOTBLOQ: Ecosistema integral para el diseño, fabricación y programación de robots DIY}\\[0.5cm] % Title of your document

\textsc Proyecto Financiado por el Centro de Desarrollo Tecnológico Industrial (CDTI) \\
\textsc{\small \emph{EXPEDIENTE:} IDI-20150289} \\
\textsc Cofinanciado por el Fondo Europeo de Desarrollo Regional (FEDER) a través del Programa Operativo Plurirregional de Crecimiento Inteligente 2014-2020 \\[0.5cm]
{\small \emph{ACRÓNIMO DEL PROYECTO:} BOTBLOQ}\\[0.5cm] % Minor heading such as course title

\includegraphics[width=7.5cm]{CDTI}
\includegraphics[width=7cm]{FEDER}\\[1.0cm]

\textsc{\Large ENTREGABLE E.2.1. Especificación de requisitos de la aplicación a desarrollar identificando las habilidades.}\\[0.5cm] % Major heading such as course name

%Resumen del documento
\HRule \\[0.4cm]
\renewcommand{\abstractname}{RESUMEN DEL DOCUMENTO - CAMBIO}
\begin{abstract}
Your abstract goes here...
...
\end{abstract}
\HRule \\[0.4cm]

{\large \today}\\[3cm] % Date, change the \today to a set date if you want to be precise


\vfill % Fill the rest of the page with whitespace

\end{titlepage}

%----------------------------------------------------------------------------------------
%	TABLE OF CONTENTS
%----------------------------------------------------------------------------------------

\tableofcontents % Include a table of contents

\newpage % Begins the essay on a new page instead of on the same page as the table of contents 

%----------------------------------------------------------------------------------------
%	INTRODUCTION
%----------------------------------------------------------------------------------------




%----------------------------------------------------------------------------------------

\end{document}